\documentclass{beamer}
\usepackage{amssymb}
\usepackage{amsmath}
\usepackage[french]{babel}
\DeclareMathAlphabet{\mathmybb}{U}{bbold}{m}{n}
\newcommand{\1}{\mathmybb{1}}
\title{La formule d'Itô}

\date{\today}
\author{Lucas Lejeune, BA3-MATH-I}

\usetheme{Madrid}

\newcommand{\indep}{\perp \!\!\! \perp}
\AtBeginSection[]{
  \begin{frame}
  \vfill
  \centering
  \begin{beamercolorbox}[sep=8pt,center,shadow=true,rounded=true]{title}
    \usebeamerfont{title}\insertsectionhead\par%
  \end{beamercolorbox}
  \vfill
  \end{frame}
}
\begin{document}

\frame{\titlepage}

\begin{frame}{Table des matières}
  % \frametitle{Summary}
  \begin{enumerate}
    \item Option Call
    \item Mouvement brownien et processus stochastique
    \item Modèle de Black-Scholes
    \item Retour au problème
    \item Intégrale d'Itô
    \item Formule d'Itô
    \item Résolution de l'équation de Black-Scholes
  \end{enumerate}
\end{frame}
\section{Option Call}
\begin{frame}{Option Call}
  \begin{block}{Definition}
    Une \textbf{option call} est un produit dérivé, contrat entre deux parties, qui donne à l'acheteur le droit (le vendeur est en revanche tenu de se plier à la décision de l'acheteur) d'acheter une quantité donnée d'un actif sous-jacent à un prix précisé à l'avance (ce prix est appelé le \textbf{strike}) (source: wikipedia.org)
  \end{block}
  \pause
  \begin{itemize}
  \item On ne s'intéressera ici qu'aux options ayant une date d'échéance donnée, et où il n'est possible de les utiliser que le jour de la date d'échéance.
  \item Dans la suite, on notera K le \textbf{strike}.
  \item Enfin, on définit le \textbf{payoff} comme étant le rendement intrinsèque d'une option
    \item Une option call ne donne pas l'obligation à l'acheteur de faire valoir son option.
  \end{itemize}
\end{frame}

\begin{frame}{Option Call}
  \includegraphics[width=12cm]{imgs/strike.png}
\end{frame}
\begin{frame}{Option Call}
  \includegraphics[width=12cm]{imgs/nostrike.png}
\end{frame}
\begin{frame}{Option Call}{Formule du pricing}
\begin{block}{Théorème fondamental du Pricing}
    Si $X_T$  représente le prix de l'actif sous-jacent au temps $T$, et si $r$ représente le taux d'intérêt, on voit bien que la valeur de l'option call sera de
    \begin{equation} \label{tfp}
      \underbrace{e^{-Tr}}_{\text{fact. d'actualisation}}\mathbb{E}\left[ \max \left(X_{T} - K; 0 \right) \right]
    \end{equation}
  \end{block}
\end{frame}
\begin{frame}{Option Call}{Actualisation}
  Mettons nous dans une situation où j'achète une option effective dans 3 ans à 100€ et imaginons que l'on ne prenne pas en compte l'actualisation et donnons nous un taux d'intêrêt à 2\% par an
  \pause
  \begin{center}
    \includegraphics[width=6cm]{imgs/interet.png}
    \end{center}
  \pause
  La banque gagne donc environ 6€ sur mon achat
\end{frame}
\begin{frame}{Option Call}
  \includegraphics[width=12cm]{imgs/strike.png}
\end{frame}
\section{Processus stochastiques et mouvement brownien}
\begin{frame}{Processus Stochastique}
  \begin{block}{Définition}
    On appelle processus stochastique la donnée
    \begin{equation}
      X = (\varOmega, \mathcal{F}, \left(  X_{t} \right)_{t\in T}, \mathbb{P})
    \end{equation}
    où $ \varOmega $ est un ensemble, $ \mathcal{F} $ est une $\sigma$-algèbre sur $ \varOmega $, $\mathbb{P}$ est une mesure de probabilité sur $ \left( \varOmega , \mathcal{F} \right)$ et $T \subset \mathbb{R}^{+}$ (représente le temps).
    Enfin, $\left( X_{t} \right)_{t\in T} $  est une famille de variable aléatoire indexée par $ T $.
  \end{block}
\end{frame}
\begin{frame}{Mouvement Brownien}
  \begin{block}{Définition}
    Un processus $ B = (\varOmega, \mathcal{F}, \left(  X_{t} \right)_{t\in T}, \mathbb{P} ) $ à valeurs réelles est un mouvement brownien si
    \begin{subequations}
      \begin{equation} B_{0} = 0 \qquad \mathbb{P}-p.s \end{equation}
      \begin{equation} \forall s \in \left[0, t\right], \qquad B_{t} - B_{s} \indep \mathcal{F}_{s} \end{equation}
      \begin{equation} \forall s \in \left[0, t\right], \qquad B_{t} - B_{s} \sim \mathcal{N} \left( 0, t-s\right)\end{equation}
      \end{subequations}
    \end{block}
    Le mouvement brownien est donc un type particulier de {\em marche aléatoire}.
\end{frame}
% \begin{frame}{Explication et représentation du problème}
%   \includegraphics<2>[width=7.5cm]{Kyotoimgs/overunder.png}
%   \hfill
%   \hfill \includegraphics<2>[width=7.5cm]{imgs/under.png}
% \end{frame}
\section{Modèle de Black-Scholes}
\begin{frame}{Modèle de Black-Scholes}{Mouvement Brownien Arithmétique}
  \begin{block}{Définition}
    Un processus brownien arihtmétique est un processus de la forme
    \begin{equation}
      dX_{t} = \mu dt + \sigma dB_{t}
    \end{equation}
    où $\mu$ représente le drift et $\sigma$ représente la variance.
    \end{block}
  \end{frame}
  \begin{frame}{Modèle de Black-Scholes}{Influence des différents paramètres dans le mouvement brownien arithmétique}

    \end{frame}
\begin{frame}{Modèle de Black-Scholes}
  \begin{block}{Définition}
    On va choisir de modéliser le marché en utilisant le processus stochastique satisfaisant l'équation stochastique suivante pour $M > 0$
    \begin{equation}
      \begin{cases}
        dX_{t} = \mu X_{t} dt + \sigma X_{t} dB_{t} \\
        X_{0} = M
      \end{cases}
      \end{equation}
      est l'équation de Black-Scholes. \\
    \end{block}
    \pause
    \begin{alertblock}{}
    Le but va donc être de comprendre ce processus de manière plus globale, il nous faut donc une intégrale de la forme
    \[
      X_{T} - X_{0} = \int_0^{T} \mu X_{t} dt + \int_{0}^{T} \sigma X_{T} dB_{t}
    \]
  \end{alertblock}
  \pause
  Le but va donc maintenant être de définir cette intégrale
\end{frame}
 \begin{frame}{Représentation de cinq $X_{t}$ résolvant Black-Scholes}{$\mu =0.1, \sigma =0.3$}
   \includegraphics[width=12cm]{imgs/bs5.png}
 \end{frame}
 \begin{frame}{Représentation de cinq $X_{t}$ résolvant Black-Scholes}{$\mu =2, \sigma =0.3$}
   \includegraphics[width=12cm]{imgs/mu2.png}
 \end{frame}
 \section{Construction de l'intégrale d'Itô}
 \begin{frame}{Construction de l'intégrale d'Itô}
   \begin{block}{Définition}<1->
     On définit $\mathcal{H}^{2}$ comme étant l'espace des fonctions mesurables et adaptées respectant la condition d'intégrabilité \\
     \[
       \mathbb{E}\left[\int_{0}^{T}f^{2}(\omega, t) dt \right] < \infty
     \]
   \end{block}
   \begin{block}{Propriétés de base de l'intégrale}
     Notre intégrale devrait être définie sur $\mathcal{H}^{2}$, celle-ci devrait être un opérateur linéaire, de plus
\iffalse [ [ \fi  si $ f(\omega, t) = \1_{\left] a; b \right]} $ on aimerait
     \[
       I(f)(\omega) = \int_{a}^{b}f(\omega, t) dB_{t} = B_{b} - B_{a}
     \]
   \end{block}
 \end{frame}
 
 \begin{frame}{Définition de l'intégrale d'Itô sur $\mathcal{H}^2_0$}
      \begin{block}{Définition}
     On définit maintenant $\mathcal{H}^{2}_{0}$ comme le sous-ensemble des fonctions dans $\mathcal{H}^{2}$ telles que celles-ci soient de la forme \iffalse [[ \fi
     \[
       f(\omega, t) = \sum_{i=0}^{n-1}a_{i}(\omega) \1_{\left] t_{i} ; t_{i+1} \right] }
     \]
     \end{block}
     \begin{block}{Définition}<2->
       Soit $ f \in \mathcal{H}^{2}_{0} $ on va définir l'intégrale d'Itô comme étant
       \[
         I(f)(\omega) = \sum_{i=0}^{n-1} a_{i}(\omega)\left( B_{t_{i+1}} - B_{t_{i}} \right)
       \]
     \end{block}
     \pause
   En fait, comme $ \mathcal{H}^{2}_{0}$ est dense dans $\mathcal{H}^{2}$, on peut définir l'intégrale sur tout $\mathcal{H}^2$
 \end{frame}
 \section{Lemme d'Itô}
   \begin{frame}{Lemme d'Itô}{Cas simple}
     \begin{block}{Lemme d'Itô}
       Avec $f: \mathbb{R} \rightarrow \mathbb{R} $ de classe $\mathcal{C}^{2} $, on a
       \begin{equation}
         f(B_{t}) = f(0) + \int_{0}^{t}f'(B_{S})dB_{S} + \frac{1}{2}\int_{0}^{t} f''(B_{s})ds
       \end{equation}
     \end{block}
     \pause
     On note la similarité entre cette équation et le théorème fondamental de l'analyse, à ceci près qu'une intégrale de la dérivée seconde de f s'invite dans l'équation
     \begin{block}{Lemme d'Itô, avec plusieurs variables}
       Soit $f\in \mathcal{C}^{1, 2}(\mathbb{R}^{+} \times \mathbb{R})$, on a
       \begin{equation}
           f(t, B_{t}) = f(0, 0) + \int_{0}^{t}\frac{\partial f}{\partial x}(s, B_{s}) dB_{s} + \int_{0}^{t}\frac{\partial f}{\partial t}(s, B_{s}) ds + \frac{1}{2} \int_{0}^{t}\frac{\partial ^{2} f}{\partial x^{2}}(s, B_{s})
         \end{equation}
         \end{block}
     \end{frame}
     \begin{frame}{Lemme d'Itô, notation Shorthand}
       \begin{block}{Lemme d'Itô (notation)}
         Soit $f\in \mathcal{C}^{1, 2}(\mathbb{R}^{+} \times \mathbb{R})$, et $ X_{t} = f(t, B_{t})$ un processus stochastique, on écrira
         \begin{equation}
           dX_{t} = \frac{\partial f}{\partial x}(t, B_{t}) dB_{t} + \frac{\partial f}{\partial t}(t, B_{t}) dt + \frac{1}{2} \frac{\partial^{2}f}{\partial x^{2}}(t, B_{t}) dt
         \end{equation}
       \end{block}
     \end{frame}
     \begin{frame}{Lemme d'Itô}{Box-Calculus}
       \begin{block}{Box-Calculus}
         \begin{center}
           \begin{tabular}{|c|c|c|}
             \hline
             $\cdot$ & $dt$ & $dB_t$ \\
             \hline
             $dt$ & 0 & 0 \\
             $dB_{t}$ & 0 & $dt$ \\
             \hline
           \end{tabular}
         \end{center}
         \end{block}
       \end{frame}
   \begin{frame}{Lemme d'Itô}{Application à Black-Scholes}
     On reprend notre équation stochastique de Black-Scholes
     \begin{equation}
       dX_{t} = \mu X_{t} dt + \sigma X_{t} dB_{s}
     \end{equation}
     \pause
     et on s'intéresse à $d \ln X_{t}$ en y appliquant la formule d'Itô
     \[
       d \ln X_{t} = \frac{1}{X_{t}} \left( \mu X_{t} dt + \sigma X_{t} dB_{t} \right) - \frac{1}{2X_{t}^{2}} \left( \mu X_{t} dt + \sigma X_{t} dB_{t} \right) \left( \mu X_{t}dt + \sigma X_{t} dB_{t}\right)
     \]
     \pause
     ensuite, en appliquant le Box-Calculus, on voit qu'il est possible de simplifier cette dernière expression
     \begin{equation}
       d \ln X_{t} =  \mu dt + \sigma dB_{t} - \frac{\sigma^{2}}{2} dt
     \end{equation}
   \end{frame}
   \begin{frame}{Lemme d'Itô}{Application à Black-Scholes (suite)}
     \[
       d \ln X_{t} = \left( \mu - \frac{\sigma^{2}}{2} \right) dt + \sigma dB_{t}
     \]
     \pause
     On intègre ensuite de chaque côté de l'équation avec l'intégrale d'Itô
     \[
       \int_{0}^{T} d \ln \left(  X_{t} \right) = \int_{0}^{T} \left(\mu - \frac{\sigma^{2}}{2} \right) dt + \int_{0}^{T} \sigma dB_{t}
     \]
     \pause
     \[
       \ln \left(\frac{X_{t}}{X_{0}} \right) = \left( \mu - \frac{\sigma^{2}}{2} \right) T + \sigma B_{T}
     \]
     \pause
     \begin{equation} \label{eq:9}
       X_{t} = \exp \left(  \left( \mu - \frac{\sigma^{2}}{2} \right) T + \sigma B_{T} \right)
     \end{equation}
   \end{frame}
   \begin{frame}{Lemme d'Itô}{Déduction de la formule de Black-Scholes}
     \begin{block}{Définition}
       Soit $Z \sim \mathcal{N}\left(0, 1 \right)$, et soient $\mu \in \mathbb{R}$ et $\sigma \in \mathbb{R}^{+}$, alors la variable définie par $ X = e^{\mu + \sigma Z}$ suit une loi log-normale.
     \end{block}
     On reprend donc $ \ref{eq:9} $ pour voir que $X_{T} \sim \log \mathcal{N} \left( \left( \mu - \frac{\sigma^{2}}{2} \right) T; \sigma \sqrt{T}  \right)$ \\
     \pause
     Tout ce qu'il reste à faire est alors de calculer $\mathbb{E}\left[\max(X_{t} - K; 0)e^{-tr}\right]$ où $ X_{t}$ suit une log-normale.

   \end{frame}
   \begin{frame}{Formule de Black-Scholes}
     Nos calculs précédents nous donnent finalement la formule de Black-Scholes telle qu'étudiée en actuariat
     \begin{block}{Théorème}
       Dans le modèle de Black-Scholes, la valeur d'une option call est donné par la formule
       \begin{equation}
         X_{0}\varPhi(d_{1}) - Ke^{-rT}\varPhi(d_{2})
       \end{equation}
       où
       \[
         d_{1} = \frac{1}{\sigma \sqrt{T}} \left( \ln \left(\frac{X_{0}}{K} \right) + \left( r + \frac{\sigma^{2}}{2} \right)T \right)
       \]
       et
       \[
         d_{2} = d_{1} - \sigma \sqrt{T}
       \]
    \end{block}
  \end{frame}
  \begin{frame}{Merci!}
    \includegraphics[width=12cm]{imgs/100b.png}
  \end{frame}
\end{document}
